\chapter{引论}
\thispagestyle{empty}

欢迎来到为傻瓜设计的音乐理论课程。当你听到音乐理论这个词语的时候,你想到的是什么?是否你小学时候的音乐老师在钢琴后面对你怒目而视的场景会出现在你的脑海中呢?或者是另外一个画面,音乐学院的学生们在理论课上尽力地记下特雷门琴的口哨声。如果这两个观点中的任何一个与你自己对音乐理论是什么的看法很接近,那么这本书有希望是一个令人愉快的惊喜。

对于许多自学成才的音乐家来说,关于理论的想法可能是令人畏惧的,甚至有一点自我挫败。毕竟,如果你已经可以阅读吉他谱并能弹奏一些曲调的话,你为什么要把你已经知道的东西和理论扯上关系?

即使是最基础的音乐理论的训练都可以给你拓展作为一个音乐人范围和能力的钥匙。相当数量的音符阅读能力将使得你可以演奏经典的钢琴音乐,而关于和弦进行的基本知识可以让你在创作自己的音乐的道路上启航。

\textbf{关于此书}

\texttt{Music Theory For Dummies}被用来设计教授你想流畅地打击出坚实的节拍,阅读乐谱,学着预测一首曲子的走向,不管你是在阅读别人的音乐还是编写自己的音乐的时候所需要知道的一切东西。

每一章的内容都被设计地相对独立,你没有必要需要阅读每一章以知道它的下一章在讲什么。不过这确实有帮助,因为音乐的概念都是由简单到复杂的。为了找到你所需要的信息,你可以使用目录作为参考点,或者你可以翻一下书后面的索引。

\textbf{这本书为谁设计的}

这本书是为所有类型的音乐家设计的,从绝对的入门者到那些从未学会即兴创作的经典乐学生,到知道如何将音乐组合在一起,而从来不花费精力找出如何从吉他谱和铅片之外阅读音乐的经验丰富的音乐家。

\texttt{完全的入门者}

我们写这本书的目的是,它将伴随着初出茅庐的音乐家从最初的步骤到音符的阅读和节奏的敲打,一直到运用音乐理论的原理创作音乐的第一次真正尝试。入门的音乐人应该从本书的开头开始,从第一部分开始并保持前进。这本书被组织成音乐学院的音乐理论课程将提供给你的教学计划,取决于你是一个多快的学习者。

\texttt{退学的音乐系学生}

这本书也是为那些小时候上过音乐课的音乐家写的,他们还记得如何阅读乐谱,但是从未接触过构建音阶的原理,基本的即兴创作以及如何与其他的音乐家打成一片的音乐家。有很多这样的人在那里,这本书的目的是轻轻地让你回到演奏音乐的乐趣。它向你展示了如何在音乐演奏的限制之外工作,真正开始即兴创作,甚至创作自己的音乐。

\texttt{有经验的表演者}

\texttt{Music Theory For Dummies}也适用于那些已经知道如何演奏音乐


	
	
	
	
	
	
	
	
	
	
	
	
	
	
	
	
	
	
	
	
	
	
	
	
	
	

	
	
	
	
	
	
	
	
	
	
	
	
	
	
	
	
	
	
	
	
	

%\setlength{\fboxrule}{0pt}\setlength{\fboxsep}{0cm}
%\noindent\shadowbox{
%\begin{tcolorbox}[arc=0mm,colback=lightblue,colframe=darkblue,title=学习目标与要求]
%\kai\textcolor{darkblue}{1.~~了解科学计算的一般过程.}\\
%\kai\textcolor{darkblue}{2.~~了解数值计算方法的研究内容和特点.}\\
%\kai\textcolor{darkblue}{3.~~理解数值计算误差的有关概念.}\\
%\kai\textcolor{darkblue}{4.~~掌握数值计算误差的控制方法.}
%\end{tcolorbox}}
%\setlength{\fboxrule}{1pt}\setlength{\fboxsep}{4pt}
%
%
%\section{Colored boxes}
%
%\begin{tcolorbox}[colback=red!5,colframe=red!75!black]
%  My box.
%\end{tcolorbox}
%
%\begin{tcolorbox}[colback=blue!5,colframe=blue!75!black,title=My title]
%  My box with my title.
%\end{tcolorbox}
%
%\begin{tcolorbox}[colback=green!5,colframe=green!75!black]
%  Upper part of my box.
%  \tcblower
%  Lower part of my box.
%\end{tcolorbox}
%
%\begin{tcolorbox}[colback=yellow!5,colframe=yellow!75!black,title=My title]
%  I can do this also with a title.
%  \tcblower
%  Lower part of my box.
%\end{tcolorbox}
%
%\begin{tcolorbox}[colback=yellow!10,colframe=red!75!black,lowerbox=invisible,
%  savelowerto=\jobname_ex.tex]
%  Now, we play hide and seek. Where is the lower part?
%  \tcblower
%  I'm invisible until you find me.
%\end{tcolorbox}
%
%\begin{tcolorbox}[colback=yellow!10,colframe=red!75!black,title=Here I am]
%  \input{\jobname_ex.tex}
%\end{tcolorbox}
%
%
%\begin{tcolorbox}[colback=blue!50,colframe=blue!25!black,coltext=yellow,
%    fontupper=\Large\bfseries,arc=6mm,boxrule=2mm,boxsep=5mm]
%  ofFunny settings.
%\end{tcolorbox}
%
%\subsection{\LaTeX-Table}
%
%\begin{table}[h]\begin{center}\color{darkblue}\caption{计算结果}\color{black}\label{tab1-2}
%{\footnotesize
%\begin{tabular}{r|r||r|r||r|r||r|r}\arrayrulecolor{darkblue}\hline\rowcolor{lightblue}
%  $n$&$I_n$&$n$&$I_n$&$n$&$I_n$&$n$&$I_n$\\\hline
%  19&0.008\ 3&14&0.011\ 2&9&0.016\ 9&4&0.034\ 3\\
%  18&0.008\ 9&13&0.012\ 0&8&0.018\ 8&3&0.043\ 1\\
%  17&0.009\ 3&12&0.013\ 0&7&0.021\ 2&2&0.058\ 0\\
%  16&0.009\ 9&11&0.014\ 1&6&0.024\ 3&1&0.088\ 4\\
%  15&0.010\ 5&10&0.015\ 4&5&0.028\ 5&0&0.182\ 3\\\hline
% \end{tabular}}\end{center}\end{table}
%
%
%\section{\LaTeX-Examples}
%
%\begin{tcblisting}{colback=red!5,colframe=red!75!black}
%This is a \LaTeX\ example:
%$\displaystyle\sum\limits_{i=1}^n i = \frac{n(n+1)}{2}$.
%\end{tcblisting}
%
%
%\section{Theorems}
%
%\begin{defi}{Summation of Numbers}{defi1.1}
%  For all natural number $n$ it holds:\\[2mm]
%  $\displaystyle\sum\limits_{i=1}^n i = \frac{n(n+1)}{2}$.
%\end{defi}
%
%\begin{theo}{Summation of Numbers}{theo1.1}
%  For all natural number $n$ it holds:\\[2mm]
%  $\displaystyle\sum\limits_{i=1}^n i = \frac{n(n+1)}{2}$.
%\end{theo}
%
%\begin{coro}{Summation of Numbers}{coro1.1}
%  For all natural number $n$ it holds:\\[2mm]
%  $\displaystyle\sum\limits_{i=1}^n i = \frac{n(n+1)}{2}$.
%\end{coro}
%We have given Theorem \ref{Theorem:theo1.1} on page \pageref{Theorem:theo1.1}.
%
%
%
%\begin{table}[h]\begin{center}\color{darkblue}\caption{计算结果}\color{black}\label{tab1-1}
%{\footnotesize
%\begin{tabular}{r|r||r|r||r|r||r|r}\arrayrulecolor{darkblue}\hline\rowcolor{lightblue}
%  $n$&$I_n$&$n$&$I_n$&$n$&$I_n$&$n$&$I_n$\\\hline
%  1&0.088\ 4&6&0.034\ 4&11&-31.392\ 5&16&9.814\ 5e+4\\
%  2&0.581\ 0&7&-0.029\ 0&12&157.045\ 7&17&-4.907\ 3e+5\\
%  3&0.043\ 1&8&0.270\ 1&13&-785.151\ 6&18&2.453\ 6e+6\\
%  4&0.347\ 0&9&-1.239\ 3&14&3.925\ 8e+3&19&-1.226\ 8e+7\\
%  5&0.026\ 5&10&0.296\ 7&15&-1.962\ 9e+4&20&6.134\ 1e+7\\\hline
%\end{tabular}}\end{center}\end{table}
%
%\section{graphicx}
%
%\begin{figure}[h]
%\begin{minipage}[t]{0.5\linewidth}
%\centering
%%\includegraphics[totalheight=1.2in]{fig/tu2-2}
%\caption{不动点迭代法收敛} \label{fig:tu2-2}
%\end{minipage}
%\begin{minipage}[t]{0.5\linewidth}
%\centering
%%\includegraphics[totalheight=1.3in]{fig/tu2-3}
%\caption{不动点迭代法发散} \label{fig:tu2-3}
%\end{minipage}
%\end{figure}
%
%
%
%\vspace{0.5cm}
%\addcontentsline{toc}{section}{\protect\numberline{}{习题一}}
%\markboth{习题一}{习题一} \centerline{\textcolor{darkblue}{\hei\zihao{4}
% 习题一}}\vspace{0.5cm}
%
%
